\section{Problem 1 – Logic Circuit Design \textit{(send help)}}

\paragraph{Question 1}
\emph{Joint effort.}
Thanks to Igor who first brute-forced the solution for $n=3$.
Takato did the same but using \emph{k-maps} for each $M_i$.

Taku first draw a table for $n=2$:   

\begin{table}[!h]
\centering
\begin{tabular}{cc|cccc}
$K_1$ & $K_0$ & $M_3$ & $M_2$ & $M_1$ & $M_0$ \\
\hline
0  & 0  & 0  & 0 & 0 & 1 \\
0  & 1  & 0  & 0 & 1 & 1 \\
1  & 0  & 0  & 1 & 1 & 1 \\
1  & 1  & 1  & 1 & 1 & 1
\end{tabular}
\end{table}
And then for $n=3$:
\begin{table}[!h]
\centering
\begin{tabular}{lll|llll|llll}
$K_2$ & $K_1$ & $K_0$ & $M_7$ & $M_6$ & $M_5$ & $M_4$ & $M_3$ & $M_2$ & $M_1$ & $M_0$ \\
\hline
0  & 0  & 0  &    &    &    &    &    &    &    & 1  \\
0  & 0  & 1  &    &    &    &    &    &    & 1  & 1  \\
0  & 1  & 0  &    &    &    &    &    & 1  & 1  & 1  \\
0  & 1  & 1  &    &    &    &    & 1  & 1  & 1  & 1  \\
\hline
1  & 0  & 0  &    &    &    & 1  & 1  & 1  & 1  & 1  \\
1  & 0  & 1  &    &    & 1  & 1  & 1  & 1  & 1  & 1  \\
1  & 1  & 0  &    & 1  & 1  & 1  & 1  & 1  & 1  & 1  \\
1  & 1  & 1  & 1  & 1  & 1  & 1  & 1  & 1  & 1  & 1 
\end{tabular}
\end{table}

Notice the repetitive pattern above.
Let's build solution for $n=3$ using $n=2$.
A circuit for $n=2$ is easy to build: let it have inputs $K_1^{(2)}, K_0^{(2)}$ and outputs $M_i^{(2)}$ for $i=0,1,2,3$.
Let the solution for $n=3$ have inputs $K_j^{(3)}$ for $j=0,1,2$ and outputs $M_l^{(3)}$ for $l=0,1,\cdots7$.
Now set:
\begin{align*}
    K_0^{(2)} &:= K_0^{(3)} \\
    K_1^{(2)} &:= K_1^{(3)} \\
    M_i^{(3)} &:= M_i^{(2)} \lor K_2^{(3)}  && \text{for $i=0,1,2,3$} \\
    M_i^{(3)} &:= M_i^{(2)} \land K_2^{(3)}  && \text{for $i=4,5,6,7$}
\end{align*}

This way we can generalize: to build solution for $n+1$, we can reuse solution for $n$ with addition of $2^n$ more gates.
More specifically, solution for $n$ has $2^n$ outputs.
We \emph{or} first 'half' of output wires with $K_{n+1}$, and \emph{and} the rest.

With this recursive definition, we can easily compute $S(n)$ \textit{size} of the circuit with $n$ inputs:
\begin{align*}
    S(n) = S(n-1) + 2^{n} = \sum_{i=0}^{i=n} 2^i = 2^{n+1} - 1 = O(N)
\end{align*}

Let $D(n)$ denote \textit{depth} of the circuit with $n$ inputs. Recall that $N=2^n$ stands for number of outputs.
\begin{align*}
    D(n) = D(n-1) + 1 = O(n) = O(\log N)
\end{align*}

Don't forget to add a switch $Z$ at the end of the final answer:
simply \emph{and} every $M_i$ with negated $Z$.
It adds $1$ to depth and $2^n$ to size.
Although, it doesn't change asymptotic size and depth, respectively.


\paragraph{Question 2}
(send help) Limitation on depth makes it difficult.

\section{Problem 2 – Bucket Sort}

Setting of the problem: we got $N$ element and $K$ buckets.

\paragraph{Question 1}
\begin{verbatim}
B[m].get(j)
\end{verbatim}

\paragraph{Question 2}
The input sequence might be in an \textbf{increasing} order and all elements might fall into one bucket.
Thus, line $6$ will be executed:
\begin{equation*}
    C = 0 + 1 + \cdots + (n-1) = \frac{n(n-1)}{2}
\end{equation*}
times.
Note that it has to be an increasing order, because items are being inserted at the beginning of a bucket.
If we wanna do evil, we must make every item traverse the whole bucket, until the end.
It is possible when every inserted item is bigger than any element in the bucket, that is, input sequence is of increasing order.


\paragraph{Question 3}
Let $n_i$ denote size of $i$-th bucket.
Line $6$ is de facto an \emph{insertion sort}, which means that $i$-th~bucket will be sorted in $O(n^2)$.
Total running time will be  to:
\begin{equation*}
    C = \sum_{i=1}^{K} O(n_i^2)
\end{equation*}
Taking expectation:
\begin{equation*}
    \mathbb{E}(C) = \mathbb{E}[\sum_{i=1}^{K} O(n_i^2)]
                  = \sum_{i=1}^{K} \mathbb{E}[O(n_i^2)]
                  = \sum_{i=1}^{K} O(\mathbb{E}[n_i^2])
\end{equation*}
What is $[\mathbb{E}n_i^2]$? 
Note that, by definition $Var(X) = \mathbb{E}[X^2] - (\mathbb{E}X)^2$ for any random variable $X$.
Let $X_{i,j}$ be an indicator random variable:
\begin{equation*}
    X_{i,j} = \begin{cases}
    1 & \text{element $j$ went to bucket $i$} \\
    0 &  \text{otherwise}
    \end{cases}
\end{equation*}
\begin{equation*}
    n_i = \sum_{j=1}^{N} X_{i, j}
\end{equation*}
Since we have $K$ buckets and every of them is equally likely, probability of "going to bucket $i$" is $\frac{1}{K}$.
Thus, expectation of $n_i$ is:
\begin{equation*}
     \mathbb{E}(n_i) = \sum_{j=1}^{N} \mathbb{E}(X_{i, j}) = \frac{N}{K}
\end{equation*}

We can also notice, that, $n_i$ is just a binomial random variable with expectation $np$ and variance $np(1-p)$.
Here a trial is mapping an item into bucket, and the success is placing it into $i$-th bucket.
There are $n=N$ trials and probability of success if $\frac{1}{K}$.
\begin{align*}
    \mathbb{E}[n_i^2] = Var[n_i] + (\mathbb{E}[n_i])^2 &= N\frac{1}{K}(1-\frac{1}{K}) + (\frac{N}{K})^2 \\
    &= \frac{N^2 + NK - N}{K^2}
\end{align*}
Finally:
\begin{align*}
    \mathbb{E}[C]   = O(\sum_{i=1}^{K}\mathbb{E}[n_i^2]) 
                    = O(\sum_{i=1}^{K} \frac{N^2 + NK - N}{K^2})
                    = O(\frac{N^2}{K} + N)
\end{align*}

\paragraph{Question 3}
The following contribute to total expected running time:
\begin{itemize}
    \item finding a bucket for each element $O(N)$
    \item sorting each bucket, $O(\frac{N^2}{K} + N)$
    \item for each bucket, getting its content: $O(K\frac{N}{K}) = O(N)$
\end{itemize}
Total running time:
\begin{equation*}
    O(\frac{N^2}{K} + 3N)
\end{equation*}
When $K$ is $O(N)$, then we get $O(N)$ expected running time.
When $K$ is $O(1)$, then the running time is $O(N^2)$.

\paragraph{Question 4}
\begin{itemize}
    \item BS is stable, QS isn't
    \item BS isn't in-place, QS is
    \item BS runs $O(N)$ expected, QS is $O(NlogN)$
    \item BS has upper limit on keys, QS hasn't
\end{itemize}




\section{Problem 3 – Automata Theory}
\paragraph{Question 1}
\begin{verbatim}
--> o --a--> o --a--> o --b--> (o)
   / \      / \      / \       / \
   \ /      \ /      \ /       \ /
    E        E        E         E
\end{verbatim}


\paragraph{Question 2}
Write
\begin{equation*}
    L_2 = \{w \in \Sigma^* | v \preceq w \text{ for some } v \in L \}
\end{equation*}
where $v \preceq w$ indicates that $v$ is subsequence of $w$.
In other words, to in order to get a word from $L_2$, we first fix a word $v \in L$ and then intertwine its letters with some "garbage" from $\Sigma^*$.

% The first idea is to reuse Q1: for each $v \in L$, $v = v_1 v_2\cdots v_{|v|}$ we will create a "small" automaton with $v_{|v|}$ states.
% We will label transitions between next states with $v_1, v_2, \cdots, v_{|n|}$.
% Additionally, each state has a loop to itself labeled with $\Sigma^*$. 
% The last state (with incoming edge $v_{|v|}$) is an accepting state.
% The "big" automaton $A_2$ which accepts $L_2$ consists of $|L|$ "small" automatons.
% First states in every "small" automaton (i.e with edge $v_1$) make set of starting state of $A_2$.

Let's fix $v\in L$.
Let $p_1, p_2, \cdots p_{|v|}$ be sequence of states that $\mathcal{A}$ visits when reading $v$.
Label transitions between $p_i$'s with next letters of $v$.
Moreover, for each $p_i$, add a loop to itself labeled with $\Sigma$.
Make $p_1$ the start state.

\paragraph{Question 3}
First,
\begin{equation*}
    L' = \{ w \in \Sigma^* | v\in L \text{ for every } v\in \Sigma^* \text{ such that  } v\preceq w\}
\end{equation*}
is kinda quirky. 
We already supposed that $L \subseteq \Sigma^*$, so why do we need $v\in \Sigma^*$ above?

Let's start with constructing a NFA $\mathcal{N}$ which accepts L'.
Let $\mathcal{N}$ be a copy of $\mathcal{A}$, where we additionally put $\Sigma$-labeled loops on every state.
Intuitively, while traversing $\mathcal{A}$, in every state, we can have a "detour", i.e. accept some garbage.

Having constructed NFA $\mathcal{N}$, move to constructing equivalent DFA $\mathcal{D}$.
It is feasible, and subset construction tells us how to it.
Formally, define automaton $\mathcal{A}$ accepting $L$ as $\mathcal{A} = (Q, \Sigma, \delta, q_o, F)$.
NFA $\mathcal{N}$ such that $\mathcal{L}(\mathcal{N}) = L'$ is defined as:
\begin{align*}
    \mathcal{N} &= (Q, \Sigma, \delta_N, q_0, F)
\end{align*}
where
\begin{equation*}
    \delta_N(q, a) = \{q\} \cup \{\delta(q,a)\}
\end{equation*}
is a "loop" on every state.
Now, equivalent DFA $\mathcal{D}$ is obtained by subset construction:
\begin{align*}
    \mathcal{D} &= (Q_D, \Sigma, \delta_D, {q_0}, F_D) \\
    Q_D &= \mathcal{P}(Q) \\
    F_D &= \{ X\in Q_D | X\cap F \neq \emptyset\} \\
    \delta_D(X,a) &= \bigcup_{q \in X} \delta_N(q,a)
\end{align*}


\paragraph{Question 4}
Since every DFA has equivalent NFA and vice versa, I'll prove (Q3) for NFA $\mathcal{N}$.
Two steps:
\begin{itemize}
    \item $w \in L' \Rightarrow \mathcal{N} \text{ accepts } w$.
    Since $w\in L'$, then there must exits $v\in L$ such that $v\preceq w$.
    $v$ is accepted both in $\mathcal{A}$ and $\mathcal{N}$, since $\mathcal{N}$ has exactly the same states as $\mathcal{A}$.
    Then, $w$ must be accepted by $\mathcal{N}$ by following the loops on letters of $w$ which do not contribute to $v$.
    \item $\mathcal{N} \text{ accepts } w \Rightarrow w \in L'$.
    To get $v$ such that $v\preceq w$, simulate $\mathcal{N}$ on $w$, skip the loops. 
    Obviously $v\in L$, because states of $\mathcal{N}$ and $\mathcal{A}$ are the same and $v$ and $w$ will end up in the same final state of $\mathcal{N}$.
\end{itemize}



\section{Problem 4 – Linear Algebra}

\paragraph{Question 1}
Condition number $\kappa(A)$ of $A$ is:
\begin{equation*}
    \kappa(A) = ||A|| \: ||A^{-1}||
\end{equation*}
where $||\cdot ||$ denotes a norm of a matrix:
\begin{equation*}
    ||A|| = \max_{x\in \mathbb{R^N}} \frac{||Ax||}{||x||} = \max_{x: ||x||=1} ||Ax||
\end{equation*}



\paragraph{Question 2}
Not sure if OK...  \\ We know $Ax = b$ and let $r = b- A\hat{x}$:
\begin{align*}
    (A+E)\hat{x} &= b \\
    A\hat{x} + E\hat{x} &= b \\
    E\hat{x} &= b - A\hat{x} \\
    E\hat{x} &= r
\end{align*}
But how do we know if $E$ is rank $1$?


\paragraph{Question 3}
\begin{align*}
    (A+ \delta A)(x+ \delta x) = b \\
    Ax + A(\delta x) + \delta A(x+ \delta x) = b
\end{align*}
Substituting $Ax = b$:
\begin{align*}
    A(\delta x) + \delta A(x+ \delta x) &= 0 \\
    - A(\delta x) &= \delta A(x+ \delta x) \\
    -\delta x &= A^{-1} (\delta A)(x+ \delta x)
\end{align*}
because matrix $A$ is not singular, so inverse $A^{-1}$ exist.
Now, take the norm both sides.
We are assuming that $A + \sigma A$ is not singular, so it has inverse and thus a norm:
\begin{align*}
    ||\delta x || &= || A^{-1} \: [\delta A(x+ \delta x) ]|| \\
      &\leq || A^{-1} ||  \: || \delta A(x+ \delta x) || \\
      &\leq || A^{-1} ||  \: || \delta A || \: || x + \delta x ||
\end{align*}
\begin{align*}
    \frac{||\delta x ||}{ || x + \delta x || } &\leq || A^{-1} ||  \: || \delta A || \\
    &= || A^{-1} || \:  \frac{|| A^{1} || }{|| A^{1} ||} || \delta A || \\
    &= \kappa(A) \frac{||\delta A||}{||A||}
\end{align*}

\paragraph{Question 4}
When an $n$ dimensional real matrix C is singular, there is a non-zero, real vector $y$ such that $Cy = 0$. 
\textit{In particular}, we can choose a unit vector $y' = \frac{y}{||y||}$. 

Let $x \in nullspace(B)$ be a unit vector i.e. $Bx = 0$ and $||x|| = 1$.
\begin{align*}
    ||A-B|| &= ||A-B|| \: ||x|| \\
            &\geq || (A-B)x || \\
            &= ||Ax - Bx|| \\
            &= ||Ax||
\end{align*}
On the other hand:
\begin{align*}
    1 = ||x|| = || A^{-1} \:(Ax) || &\leq || A^{-1} || \: ||Ax|| \\
    ||Ax|| &\geq \frac{1}{||A^{-1}||}
\end{align*}
More on condition numbers:\newline
\url{https://blogs.mathworks.com/cleve/2017/07/17/what-is-the-condition-number-of-a-matrix/}.